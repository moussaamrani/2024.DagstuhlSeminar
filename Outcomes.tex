\subsection{Outcomes}
\label{sec:Outcome}

\TD{Reviewers often wonder what the outcome of a proposed seminar would be. 
It is in the nature of seminars that it is difficult to predict the outcome. 
However, it does help to know what you envisage as the possible outcomes. 
Keep in mind that in planning those outcomes, substantial involvement of the 
invitees beyond the seminar may be unrealistic.}

One of the primary goals of this Seminar is bring together researchers and 
practitioners from historically distinct communities, namely CPS
modellers and engineers, and specialists of machine inference methods (machine 
learning, ontologies, etc.) to explore how these novel techniques may help better
design CPS artefacts, and support engineers' workflows and activities.
%
The Seminar pursues four main goals:
\begin{enumerate}
	\item Build a joint community that would pave the way towards a more systematic
	and disciplined understanding of the various uses of machine inference approaches
	in the context of CPS engineering;
	
	\item Promote cross-fertilisation of knowledge, techniques, and supporting 
	technology and tools;
	
	\item Identify current challenges that may be leveraged or solved by 
	transposing methods and techniques from other subfields.
	
	\item Impulse an innovative, and coherent vision of why, and how to best integrate
	those approaches;
	
	\item Publish joint contributions that classify the existing practice,
	summarise the state-of-the-art and of-the-practice, propose guidelines, and 
	discuss similarities and differences, as well as core concepts.
\end{enumerate}
This Seminar represents a great opportunity to gather world-leading researchers
and industry practitioners who recently achieved new results over the past years,
and who have a deep understanding of the underlying machine-inference techniques, 
as well as of the challenges related to the complexity of CPS engineering, to
exchange novel ideas and techniques, learn from one another, and promote a 
disciplined and systematic approach for integrating these techniques into the
current practice, in order to help tackle the ever-growing challenges related
to the complexity of CPS engineering.