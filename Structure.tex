\subsection{Seminar Structure}
\label{sec:Structure}

\TD{A convincing plan for the structure of the seminar should be provided. 
We do not expect a detailed hourly plan. But information like whether there 
would be talks, discussion and breakout sessions, demos, plenary session, 
etc. would help reviewers judge whether the composition of the planned 
seminar is likely to be successful.
%
Particularly for short seminars (3 days), it would be of the essence that 
the seminar is prepared and organized in a very disciplined manner in order 
to achieve a tangible outcome.}

The Seminar will be structured around \textbf{two moments}. 
\begin{itemize}
	\item The \textbf{first two days} will focus on building momentum for converging 
	on a series of concrete topics tackled by small groups. This will rely on 
	a series of \textbf{keynotes} laying down the fundamental concepts and position
	the current state-of-the-art for each subfield, and present current ideas, 
	challenges, and main research directions. A quick session will be dedicated to
	\textbf{short presentations by participants} for presenting their research
	interests and current work aligned with the Seminar's purpose. 

	\item The \textbf{rest of the week} will be dedicated to intensive discussions
	in \textbf{groups work}: they will focus on deepening the fundamental concepts, 
	and lay down background discussions to bootstrap collaborations and publications.
\end{itemize}
The topics for group sessions will be determined during the meeting, trying to match
at best the participants' interests with the current challenges and promising 
research lines. 

Throughout the duration of the Seminar, two kinds of less cognitively-demanding 
sessions will be planned, typically in the evening, to support groups work:
\begin{itemize}
	\item \textbf{``plenary'' sessions} are intended to report on groups' progress, and 
	foster discussions and feedback from other groups, potentially helping to realign 
	the groups' upcoming work; and 
	\item \textbf{``tools demo/tutorial'' sessions} aim at presenting, in a tutorial 
	introduction-style, popular tools for handling typical tasks in each subfield.
\end{itemize}
