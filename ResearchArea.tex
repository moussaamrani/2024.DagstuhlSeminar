\subsection{Research Area}
\label{sec:ResearchArea}

\TD{Timeliness and relevance of the proposed topic are important factors in 
the evaluation. Organizers are, therefore, advised to demonstrate them in 
the proposal text.
Seminars are limited to topics that substantially involve computer 
science research. There are no further restrictions.
Interdisciplinary topics, be it in computer science and its peripheral 
areas or with completely different topics (biology, chemistry, medicine, 
physics, psychology, law, history, etc.) as well as topics with industrial 
or industry-related applications are especially welcome.
Dagstuhl is a place where you can try something unusual}

Truly complex, engineered systems, known as Cyber-Physical Systems (CPS) (but
also Cyber-Physical Systems \emph{of Systems} --- CPSoS), are increasingly 
common. In an industrial setting, they are also referred to as Industry 5.0, subsuming the Internet of Things, or Industrial Internet of Things (IIoT). 
CPSoS emerge from the networking of multi-physical (mechanical, electrical, hydraulic,
biochemical, etc.) and computational (machine learning, control, signal processing, logical inference, planning, etc.) processes, often interacting with a highly uncertain environment, 
including human actors, in a socio-economic context. 
The design, production, maintenance, evolution, and adaptation of such systems 
involves multiple stakeholders, each with their own goals, methods, and paradigms, 
working concurrently while loosely collaborating, following complex, often implicit
workflows. Individual (mechanical, electrical, network or software) engineering 
disciplines only offer partial solutions to the problem of building and evolving 
such systems. 

To date, no unifying theory nor systematic design methods, techniques and tools exist
for the engineering of such systems. These are needed to unlock the potential of a next
level of hitherto unseen complexity of managed, ever evolving, complex and resilient
Cyber-Physical Systems of Systems in a circular economy. In this vein, the
International Council on Systems Engineering (INCOSE) in their recent Systems Engineering Vision 2035 \cite{TR:InCoSe-Vision:2035} state that future
approaches need to rely on Modelling, including modern AI techniques, 
to represent the systems themselves, their 
environment, but also many of the related engineering processes. 
%following the long tradition of using models to deal with complexity in other science and engineering fields
%(such as architecture, biology, chemistry, mechanical and civil engineering, etc.). 

Multi-Paradigm Modelling (MPM) advocates to model every part and aspect of a system
explicitly (including engineering processes), at the most appropriate level(s) 
of abstraction, using the most appropriate modelling formalism(s). These models,
together with their supporting infrastructure (software design tools, but also
debuggers, simulators, analysers, etc.) exhibit \emph{essential} complexity, i.e.
complexity directly emanating from the systems that cannot be reduced further;
instead, this complexity has to be managed with the help, among other techniques,
of modelling languages. MPM promotes engineering and developing of 
modelling languages, including model transformation and simulation, as well as
studying their semantics, as an effective answer to the challenges of designing
CPSs: as a matter of fact, modelling languages are used across all the engineering
disciplines necessary for CPS to design, reason, and analyse outcomes in each
of these domains.

MPM proposes to classify, and organise how these modelling languages reflect 
the physical, computational and processing realities, as well as their relationships.
Given the complexity and heterogeneity of CPSs, a monolith approach towards
capturing, managing, and ultimately producing the many artefacts necessary in
CPS engineering has little to no chances to work. On the contrary, MPM relies 
from its very foundations on the use of multiple paradigms, eventually selecting
the most appropriate for each task: describing data, knowledge and domain understanding 
using multiple \emph{formalisms}, but also adequately capturing \emph{workflows} 
and experiments (virtual and real) required to fine-tune and calibrate models 
wrt. the reality, and making all the various CPS components adequately exchange 
and communicate.
Recently, a conceptual framework for capturing the nature of the paradigms at 
play in CPS engineering has been proposed \cite{J:Amrani-etAl:2020}, exploring some 
possible ways to combine some of them, resulting in formalisms that may reuse
existing tooling. This work targeted possible ways of \textbf{combining} 
\emph{some formalisms}, and \emph{some processes}, paving a path to a more systematic
approach in the use of \emph{multi-paradigms}.

While these kinds of combination is still not sufficiently explored, other kinds
of combinations operating at a higher level of decisions in the CPS engineering
process, require scrutiny, and therefore more research that would ultimately
lead to appropriate guidelines and tool support. Our Seminar will then focus on
the key research question:

\begin{quote}
\emph{How to effectively \textbf{combine} the various \textbf{paradigms} used in 
CPS, in order to tackle the various \textbf{sources of heterogeneity}?}
\end{quote}

\noindent
Combination as an effective solution for handling heterogeneity operates at various
level of decision. At the machine-machine level, it may promote interoperability
through co-simulation \cite{J:Gomes-etAl:2018}. At the machine-human level, 
managing efficiently data collected from physical sensors, previous experiments,




----------
M/M: computation (co-sim) + design (co-modelling)

Multi-views, multi-components
Contract-Based Design

