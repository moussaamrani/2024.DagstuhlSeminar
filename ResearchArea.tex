\subsection{Research Area}
\label{sec:ResearchArea}

\TD{Timeliness and relevance of the proposed topic are important factors in 
the evaluation. Organizers are, therefore, advised to demonstrate them in 
the proposal text.
Seminars are limited to topics that substantially involve computer 
science research. There are no further restrictions.
Interdisciplinary topics, be it in computer science and its peripheral 
areas or with completely different topics (biology, chemistry, medicine, 
physics, psychology, law, history, etc.) as well as topics with industrial 
or industry-related applications are especially welcome.
Dagstuhl is a place where you can try something unusual}

Truly complex, engineered systems, known as Cyber-Physical Systems (CPS) (but
also Cyber-Physical Systems \emph{of Systems} --- CPSoS), are becoming increasingly 
common. These are also known as Industry 4.0, Internet of Things, or  Industrial 
Internet of Things (IIoT). 
They emerge from the networking of multi-physical (mechanical, electrical, hydraulic,
biochemical, etc.) and computational (control, signal processing, logical inference, 
planning, etc.) processes, often interacting with a highly uncertain environment, 
including human actors, in a socio-economic context. 
The design, production, maintenance, evolution, and adaptation of such systems 
involves multiple stakeholders, each with their own goals, methods, and paradigms, 
working concurrently while loosely collaborating, following complex, often implicit
workflows. Individual (mechanical, electrical, network or software) engineering 
disciplines only offer partial solutions to the problem of building and evolving 
such systems. 

To date, no unifying theory nor systematic design methods, techniques and tools exist
for the engineering of such systems. These are needed to unlock the potential of a next
level of hitherto unseen complexity of managed, ever evolving, complex and resilient
Cyber-Physical Systems of Systems in a circular economy. In this vein, the
International Council on Systems Engineering (InCoSe) envisioned that future
approaches need to rely on Modelling to represent the systems themselves, their 
environment, but also many of the related engineering processes \cite{TR:InCoSe-Vision:2021}, 
following the long tradition of using models to deal with complexity in other science fields
(such as architecture, biology, chemistry, mechanical and civil engineering, etc.)

Multi-Paradigm Modelling (MPM) advocates to model every part and aspect of a system
explicitly (including engineering processes), at the most appropriate level(s) 
of abstraction, using the most appropriate modelling formalism(s). These models,
together with their supporting infrastructure (software design tools, but also
debuggers, simulators, analysers, etc.) exhibit \emph{essential} complexity, i.e.
complexity directly emanating from the systems and that cannot be reduced further;
instead, this complexity has to be managed. As a consequence, modelling languages'
engineering, including model transformation and simulation, and the study of their
semantics, are used to realise MPM, therefore promoting this approach as an 
effective answer to the challenges of designing CPSs.


M/M: computation (co-sim) + design (co-modelling)

Multi-views, multi-components
Contract-Based Design

