%%
%% This is file `sample-acmsmall.tex',
%% generated with the docstrip utility.
%%
%% The original source files were:
%%
%% samples.dtx  (with options: `acmsmall')
%% 
%% IMPORTANT NOTICE:    
%% 
%% For the copyright see the source file.
%% 
%% Any modified versions of this file must be renamed
%% with new filenames distinct from sample-acmsmall.tex.
%% 
%% For distribution of the original source see the terms
%% for copying and modification in the file samples.dtx.
%% 
%% This generated file may be distributed as long as the
%% original source files, as listed above, are part of the
%% same distribution. (The sources need not necessarily be
%% in the same archive or directory.)
%%
%% Commands for TeXCount
%TC:macro \cite [option:text,text]
%TC:macro \citep [option:text,text]
%TC:macro \citet [option:text,text]
%TC:envir table 0 1
%TC:envir table* 0 1
%TC:envir tabular [ignore] word
%TC:envir displaymath 0 word
%TC:envir math 0 word
%TC:envir comment 0 0
%%
%%
%% The first command in your LaTeX source must be the \documentclass command.
\documentclass[acmsmall]{acmart}
%% NOTE that a single column version is required for 
%% submission and peer review. This can be done by changing
%% the \doucmentclass[...]{acmart} in this template to 
%% \documentclass[manuscript,screen]{acmart}
%% 
%% To ensure 100% compatibility, please check the white list of
%% approved LaTeX packages to be used with the Master Article Template at
%% https://www.acm.org/publications/taps/whitelist-of-latex-packages 
%% before creating your document. The white list page provides 
%% information on how to submit additional LaTeX packages for 
%% review and adoption.
%% Fonts used in the template cannot be substituted; margin 
%% adjustments are not allowed.
%%
%% \BibTeX command to typeset BibTeX logo in the docs
\AtBeginDocument{%
  \providecommand\BibTeX{{%
    \normalfont B\kern-0.5em{\scshape i\kern-0.25em b}\kern-0.8em\TeX}}}

%% Rights management information.  This information is sent to you
%% when you complete the rights form.  These commands have SAMPLE
%% values in them; it is your responsibility as an author to replace
%% the commands and values with those provided to you when you
%% complete the rights form.
\setcopyright{acmcopyright}
\copyrightyear{2018}
\acmYear{2018}
\acmDOI{XXXXXXX.XXXXXXX}


%%
%% These commands are for a JOURNAL article.
\acmJournal{CSUR}
\acmVolume{56}
\acmNumber{8}
\acmArticle{25}
\acmMonth{9}

%%
%% Submission ID.
%% Use this when submitting an article to a sponsored event. You'll
%% receive a unique submission ID from the organizers
%% of the event, and this ID should be used as the parameter to this command.
%%\acmSubmissionID{123-A56-BU3}

%%
%% For managing citations, it is recommended to use bibliography
%% files in BibTeX format.
%%
%% You can then either use BibTeX with the ACM-Reference-Format style,
%% or BibLaTeX with the acmnumeric or acmauthoryear sytles, that include
%% support for advanced citation of software artefact from the
%% biblatex-software package, also separately available on CTAN.
%%
%% Look at the sample-*-biblatex.tex files for templates showcasing
%% the biblatex styles.
%%

%%
%% The majority of ACM publications use numbered citations and
%% references.  The command \citestyle{authoryear} switches to the
%% "author year" style.
%%
%% If you are preparing content for an event
%% sponsored by ACM SIGGRAPH, you must use the "author year" style of
%% citations and references.
%% Uncommenting
%% the next command will enable that style.
%%\citestyle{acmauthoryear}

\usepackage{rotating,tabularx,booktabs,siunitx,caption}
\newcolumntype{C}{>{\centering\arraybackslash}X}
\newcommand\mc[1]{\multicolumn{1}{C}{#1}} % handy shortcut macro

\usepackage{listings}
\lstset{
basicstyle=\ttfamily,
frame=single
}
\graphicspath{{img/}{pdf/}}
\usepackage[caption=false]{subfig}
\usepackage{enumitem}
\newcommand{\RQlabel}[2]{$#1.#2$}

\usepackage{multirow,booktabs,tabularx}
\usepackage{boxedminipage}
\usepackage{rotating}
\usepackage{ragged2e}
\usepackage{pdflscape}

%%%%% COMMENT SYSTEM %%%%%%%%%%%%%%%%%%%%%%%%
\newboolean{showcomments}
\setboolean{showcomments}{true} % toggle to show or hide comments
\ifthenelse{\boolean{showcomments}}{%
   \newcommand{\nb}[2]{\fcolorbox{gray}{yellow}{\bfseries\sffamily #1}{$\blacktriangleright$#2$\blacktriangleleft$}}
   \newcommand{\nbb}[1]{\fcolorbox{gray}{red}{\bfseries\sffamily #1}}
}{
   \newcommand{\nb}[2]{} 
   \newcommand{\nbb}[1]{} 
}
\newcommand\MA[1]{\nb{Moussa}{\textcolor{blue}{#1}}} 
\newcommand\AO[1]{\nb{Abdel}{\textcolor{green}{#1}}}
\newcommand\TD{\nbb{/!\textbackslash}}
%%%%%%%%%%%%%%%%%%%%%%%%%%%%%%%%%%%%%%%%


%%%%% COMMON ABBREVIATIONS %%%%%%%%%%%%%%%%%%%%%%%%
\usepackage{xspace}
\newcommand{\DSL}{\textsc{Dsl}\xspace}
\newcommand{\DSLs}{\textsc{Dsl}s\xspace}
\newcommand{\MDE}{\textsc{Mde}\xspace}

%%%%%%%%%%%%%%%%%%%%%%%%%%%%%%%%%%%%%%%%%%%%%%%%%%%




\begin{document}

\title{Animation of Model Transformations in Model-Driven Engineering: A Mapping Study}


%%
%% The "author" command and its associated commands are used to define
%% the authors and their affiliations.
%% Of note is the shared affiliation of the first two authors, and the
%% "authornote" and "authornotemark" commands
%% used to denote shared contribution to the research.
\author{Moussa Amrani}
\email{Moussa.Amrani@unamur.be}
\orcid{1234-5678-9012}
\author{Abdelkader Ouared}
\email{Abdelkader.Ouared@unamur.be}
\author{Pierre-Yves Schobbens}
\email{Pierre-Yves.Schobbens@unamur.be}
\affiliation{%
  \institution{Faculty of Computer Science / NaDI, University of Namur}
  \streetaddress{Rue Grandgagnage, 21}
  \city{Namur}
  \country{Belgium}
  \postcode{5000}
}


%%
%% By default, the full list of authors will be used in the page
%% headers. Often, this list is too long, and will overlap
%% other information printed in the page headers. This command allows
%% the author to define a more concise list
%% of authors' names for this purpose.
\renewcommand{\shortauthors}{Amrani, Ouared and Schobbens}

\input{Abstract}

%%
%% The code below is generated by the tool at http://dl.acm.org/ccs.cfm.
%% Please copy and paste the code instead of the example below.
%%
\begin{CCSXML}
<ccs2012>
<concept>
<concept_id>10002944.10011122.10002945</concept_id>
<concept_desc>General and reference~Surveys and overviews</concept_desc>
<concept_significance>500</concept_significance>
</concept>
%<concept>
%<concept_id>10011007.10011006.10011050.10011017</concept_id>
%<concept_desc>Software and its engineering~Domain specific languages</concept_desc>
%<concept_significance>500</concept_significance>
%</concept>
<concept>
<concept_id>10011007.10011006.10011050.10011058</concept_id>
<concept_desc>Software and its engineering~Visual languages</concept_desc>
<concept_significance>500</concept_significance>
</concept>
</ccs2012>
\end{CCSXML}

\ccsdesc[500]{Software and its engineering~Visual languages}
\ccsdesc[500]{General and reference~Surveys and overviews}

%%
%% Keywords. The author(s) should pick words that accurately describe
%% the work being presented. Separate the keywords with commas.
\keywords{
   Model Transformation, 
   Animation, 
   Visual Representation
}

\received{15 November 2023}
\received[revised]{TBA}
\received[accepted]{TBA}

%%
%% This command processes the author and affiliation and title
%% information and builds the first part of the formatted document.
\maketitle
\input{Introduction}
\input{Background}
\input{RM}
\input{Results}
\input{TV}
\input{Classification}
\input{RW}
%\input{Discussion}
\input{Conclusion}
\appendix
\input{Tools}



%%
%% The acknowledgments section is defined using the "acks" environment
%% (and NOT an unnumbered section). This ensures the proper
%% identification of the section in the article metadata, and the
%% consistent spelling of the heading.
%\begin{acks}
%To Robert, for the bagels and explaining CMYK and color spaces.
%\end{acks}

%%
%% The next two lines define the bibliography style to be used, and
%% the bibliography file.
\bibliographystyle{ACM-Reference-Format}
\bibliography{./CSUR}
\end{document}
